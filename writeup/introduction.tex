%!TEX root = main.tex

% =====================
% Introduction Chapter
% =====================
\chapter{Introduction}

In this thesis we present a new algorithm to solve the Vehicle Routing Problem. The Vehicle Routing Problem describes the problem of assigning and ordering geographically distributed work to a pool of resources. The aim is minimise the travel cost required to complete the work, while meeting any specified constraints, such as a maximum shift duration. Often the context used is that of a fleet of vehicles delivering goods to their customers, although the problem can equally be applied across many different industries and scenarios, and has been applied to non-logistics applications, such as microchip layout. 

Interest in the Vehicle Routing Problem has increased over the last two decades as the cost of transporting and delivering goods has become a key factor of most developed economies. Even a decrease of a few percent on transportation costs can offer a savings of billions for an economy. In the context of New Zealand (our home country) virtually every product grown, made, or used is carried on a truck at least once during its lifetime~\cite{RTFNZ}. The success of New Zealand's export industries are inextricably linked to the reliability and cost effectiveness of our road transport. Moreover, a 1\% growth in national output requires a 1.5\% increase in transport services~\cite{RTFNZ}. 

The Vehicle Routing Problem offers real benefits to transport and logistics companies. Optimisation of the planning and distribution process, such as modelled by the Vehicle Routing Problem, can offer savings of anywhere between 5\% to 20\% of the transportation costs~\cite{TV2001}. Accordingly, the Vehicle Routing Problem has been the focus of intense research since its first formal introduction in the fifties. The Vehicle Routing Problem is one of the most studied of all combinatorial optimisation problems, with hundreds of papers covering it and the family of related problems since its introduction fifty years ago.

The challenge of the Vehicle Routing Problem is that it combines two (or more, in the case some of its variants) combinatorially hard problems that are themselves known to be \nphard. Its membership in the family of \nphard\ problems makes it very unlikely that an algorithm exists, with reasonable runtime performance, that is able to solve the problem exactly. Therefore heuristic approaches must be developed to solve all but the smallest sized problems.

Many methods have been suggested for solving the Vehicle Routing Problem. In this thesis we develop a new meta-heuristic algorithm that we call the Enhanced Bees Algorithm. We adapt this from another fairly recent algorithm, named the Bees Algorithm, that was developed for solving continuous optimisation problems.

We show that the results obtained by the Enhanced Bees Algorithm are competitive with the best modern meta-heuristics available for the Vehicle Routing Problem. Additionally, the algorithm has good runtime performance, producing results within 2\% of the optimal solution within 60 seconds. This makes the Enhanced Bees Algorithm suitable for use within real world dispatch scenarios where often the dispatch process is fairly dynamic and therefore it becomes impractical for a dispatcher to wait minutes (or hours) for an optimal solution\footnote{The Enhanced Bees Algorithm was developed as part of a New Zealand Trade and Enterprise Research and Development grant for use by the dispatch software company, vWorkApp Inc.}.

\section{Content Outline}

We start in Chapter \ref{chap:background} by providing a short history of the Vehicle Routing Problem, as well as providing background material necessary for understanding the Enhanced Bees Algorithm. In particular we review the classic methods that have been brought to bear on the Vehicle Routing Problem along with the influential results achieved in the literature. In Chapter \ref{chap:pd} we provide a formal definition of the Vehicle Routing Problem and briefly describe the variant problems that have been developed in the literature. In Chapter \ref{chap:algorithm} we provide a detailed description of the Enhanced Bees Algorithm and its operation, along with a review of the objectives that the algorithm is designed to meet, and a description of how the algorithm internally represents the Vehicle Routing Problem. In Chapter \ref{chap:results} we provide a detailed breakdown of the results obtained by the Enhanced Bees Algorithm. The algorithm is tested against the well known set of test instances due to Christofides, Mingozzi and Toth~\cite{CMT:1981} and is contrasted with the other well known results from the literature. Finally, in Chapter \ref{chap:conclusion} we provide a summary of the results achieved by the Enhanced Bees Algorithm in context of the other methods available for solving the Vehicle Routing Problem from the literature. Additionally, we offer our thoughts on future directions and areas that warrant further research.  




 

