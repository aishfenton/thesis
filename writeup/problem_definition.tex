%!TEX root = main.tex

% =====================
% CHAPTER
% ---------------------
% =====================
\chapter{Problem Definition}

\begin{itemize}
	\item Full-truck-load pickup and deliveries
	\item Multiple trucks
	\item Start at depot. Return to depot
	\item Time window constraints
\end{itemize}

% =====================
% INDUSTRY 
% =====================
\section{Industry}
Full truck load (FTL) movement of freight is a common. Most bulk materials are moved as full truck loads. The classic \VRP\ problem examines the case where a truck is Less than Truck Load (\LTL). In this case a truck is loaded with many goods that are to be transported to many different customers. The truck is contained by its capacity and sometime also by the length of the trip. The \VRP\ maps well to the real-world freight business of distribution where once goods have been moved between major centres then they are distributed out to retailers or end customers.

Although in the literature \VRP\ has been extensively studied a comparatively small amount has been published on \PDP. As far as we know no papers have been published on the \PDP\ for full truck loads.     

In transporting bulk materials, such as Logging, Petroleum, Cement, ? the more common model is where the truck is filled to capacity at the primary manufacturing plant (e.g. cement plant, forestry skid site) and then distributed as a \FTL\ (or many \FTL) to a resellers such as a concrete plant or saw mill. In bulk transport the transport planner will try to minimize the amount of empty running by maximizing the amount of back-loading possible.

Empty-running is where a vehicle is traveling between jobs without any goods. For the bulk transport planner this total distance traveled isn't important as their rates are structured so that this travel is reflected in the cost of the job. A bulk transport company on the other hand isn't paid for time spent traveling between jobs.

A back-load is simply were a series of pickups and deliveries can be chained together such that empty-running time is minimized.

% =====================
% Business problem
\subsection{Operation}


% =====================
% Objectives
\subsection{Objectives}

\begin{itemize}
	\item Minimize between job travel distance
\end{itemize}

% =====================
% Constraints
\subsection{Constraints}


% =====================
% Formal definitions 
% =====================
\section{Formal definitions}


% =====================
% Sub Section
\subsection{Classic \VRP}
We formulate the \VRP\ here as an integer programming problem. Although it is possible to solve the \VRP\ as an integer programming problem this approach is only feasible for small problem sizes. Rather the problem is presented here in this fashion so that it can be stated precisely.
 
The \VRP\ is defined on a graph $(V,E)$. The vertices of the graph $V$ represent customers with $v_0$ and $v_{n+1}$ representing the depot where the vehicle starts and ends each route. Customers are denoted by $C = 1,2,...,n$. The set of edges $E$ corresponds to possible connections between customers. For our use here all connections are possible -- that is it is possible for a vehicle to drive between any two customers -- except where the depot is concerned. No edge terminates at $v_0$ and no edge originates at $v_{n+1}$ and there is no edge $\edge{0,n+1}$. Each edge $\edge{i,j} \in E$ has a corresponding cost $c_{ij}$ which typically represents the travel distance between customers.

The set of routes is denoted by $R$ and each route has a maximum capacity $q$. Each customer has a demand $d_i, i \in C$.

The model contains the decision variable $X_{ij}^r$ which is defined $\forall \edge{i,j} \in E, \forall r \in R$. $X_{ij}^r$ is equal to $1$ if route $r$ contains a trip from customer $c_i$ to customer $c_j$ and $0$ otherwise.

\begin{align}
&\intertext{Minimize:}
	&\sum_{r \in R} \sum_{\edge{ij} \in E} c_{ij}X_{ij}^r \tag{1}\label{PF:1}\\ \notag
&\intertext{Subject to:}
	&\sum_{r \in R} \sum_{j \in V} X_{ij}^r = 1 		  		&&\quad \forall i \in C \tag{2}\label{PF:2}\\
	&\sum_{i \in C} d_i \sum_{j \in C} X_{ij}^r \leq q 		&&\quad \forall r \in R \tag{3}\label{PF:3}\\
	&\sum_{j \in V} X_{0j}^r = 1 							   	&&\quad \forall r \in R \tag{4}\label{PF:4}\\
	&\sum_{j \in V} X_{j0}^r = 1 							   	&&\quad \forall r \in R \tag{5}\label{PF:5}\\
	&\sum_{i \in V} X_{ik}^r - \sum_{j \in V} X_{kj}^r = 0 &&\quad \forall k \in C \mbox{ and } \forall r \in R \tag{6}\label{PF:6}
\end{align}

The objective function \eqref{PF:1} aims to minimize costs $c_{ij}$. Constraint \eqref{PF:2} states that each customer can only be serviced by a single route. Constraint \eqref{PF:3} enforces the capacity constraint: each route can not exceed the maximum vehicle capacity. Constraint \eqref{PF:4} and \eqref{PF:5} ensure that each route starts at exactly 1 depot vertex and finishes at exactly one depot vertex. Lastly, constraint \eqref{PF:6} is a flow constraint that ensures that the number of vehicles entering a customer is equal to the number of vehicles leaving.


% =====================
% Sub Section
\subsection{\PDP}

We now similarly define the \PDP\. The \PDP\ generalizes the \VRP. Goods can now be picked up from a customer or delivered to a customer along a single route. In contrast in the \VRP\ all jobs are either picking up goods or delivery goods to a customer. The \PDP\ is defined on a graph $(V,E)$. The vertices of the graph $V$ represent all pickup and delivery locations along with $v_0$ and $v_{2n+1}$ representing the depot where the vehicle starts and ends each route. Pickup jobs are denoted by $P = \set{p_1,p_2,...,p_n}$ and delivery jobs by $D = \set{d_1,d_2,...,d_n}$. The set of edges $E$ corresponds to the possible connections between jobs. For our use here all connections are possible -- that is it is possible for a vehicle to drive between any two jobs -- except where the depot is concerned. No edge terminates at $v_0$ and no edge originates at $v_{2n+1}$ and there is no edge $\edge{0,2n+1}$. Each edge $\edge{i,j} \in E$ has a corresponding cost $c_{ij}$ which typically represents the travel distance between customers.

The set of routes is denoted by $R$ and each route has a maximum capacity $q$. Each location has a demand $d_{i} \in V$. Pickup jobs have a positive demand whereas deliveries are assumed to have a negative demand. A route containing a pickup $p_i \in P$ must also contain its corresponding delivery $d_i \in D$.

The model contains the two decision variables: $X_{ij}^r$ which is defined $\forall \edge{i,j} \in E, \forall r \in R$. $X_{ij}^r$ is equal to $1$ if route $r$ contains a trip from location $v_i$ to location $v_j$ and $0$ otherwise.

Let $y_i$ denote a an intermediate variable that stores the total load of a vehicle traveling route $r$ and visiting location $v_i$.  

\begin{align}
&\intertext{Minimize:}
	&\sum_{r \in R} \sum_{\edge{ij} \in E} c_{ij}X_{ij}^r \tag{1}\label{PF:1}\\ \notag
&\intertext{Subject to:}
% general VRP constraints
	&\sum_{r \in R} \sum_{j \in V} X_{ij}^r = 1 		  		&&\quad \forall i \in V \tag{2}\label{PF:2}\\
	&\sum_{j \in V} X_{0j}^r = 1 							   	&&\quad \forall r \in R \tag{4}\label{PF:3}\\
	&\sum_{j \in V} X_{j0}^r = 1 							   	&&\quad \forall r \in R \tag{5}\label{PF:4}\\
	&\sum_{i \in V} X_{ik}^r - \sum_{j \in V} X_{kj}^r = 0 &&\quad \forall k \in C \mbox{ and } \forall r \in R \tag{6}\label{PF:5}\\
% pickup and delivery contraints
	&p_i \in r \implies d_i \in r \forall r \in R \tag{7}\label{PF:6}\\
	&p_i \text{ appears before } d_i \in r \forall r \in R \tag{8}\label{PF:7}\\
	&X_{ij}^r = 1 \implies y_i + d_i = y_j \forall r \in R \text{ and } \forall i,j \in V \tag{9}\label{PF:8}\\
	y_0 = 0 \tag{10}\label{PF:9}\\
	&\sum_{r \in R} \sum_{j \in V} X_{ij}^r y_i \leq q		&&\quad \forall i \in V \tag{11}\label{PF:10}
\end{align}

The objective function \eqref{PF:1} aims to minimize costs $c_{ij}$. Constraint \eqref{PF:2} states that each customer can only be serviced by a single route. Constraint \eqref{PF:3} and \eqref{PF:4} ensure that each route starts at exactly 1 depot vertex and finishes at exactly one depot vertex. Constraint \eqref{PF:5} is a flow constraint that ensures that the number of vehicles entering a customer is equal to the number of vehicles leaving.

Constraint \eqref{PF:6} states that jobs pickup and delivery jobs are serviced by the same route. Constraint \eqref{PF:7} enforces that goods are picked up before being delivered. And constraints \eqref{PF:8} through \eqref{PF:10} ensure that each vehicle's capacity isn't exceeded. 

% =====================
% Sub Section
\subsection{\PDPFTL}

The mathematical model can be represented by \PDP\ above. As the demand on each job fills the truck we can simplify the model so that each pickup and delivery constraint is paired. We also add two additional constraints: vehicles fleet size is fixed and each route has a maximum distance travelled (this is used to limited the work undertaken to a shift). These aren't required for the general case but are important in most real-world applications of the problem. 

We start by using the same framework given in the \PDP\ formulation. We replace the \PDP\ concept of pickup and delivery locations, with the simpler notation of jobs $J = \set{1,2,...,n}$. Each job $j \ in J$ is represented by the arc $\edge{j_p,j_d}$ which represents the pickup and delivery locations of the job. Each job can only be serviced by a single route $r in R$.

The number of routes is set to the fixed number of vehicles we have available $|R| = k$.

% We now similarly define the \PDP\. The \PDP\ generalizes the \VRP. Goods can now be picked up from a customer or delivered to a customer along a single route. In contrast in the \VRP\ all jobs are either picking up goods or delivery goods to a customer. The \PDP\ is defined on a graph $(V,E)$. The vertices of the graph $V$ represent all pickup and delivery locations along with $v_0$ and $v_{2n+1}$ representing the depot where the vehicle starts and ends each route. Pickup jobs are denoted by $P = \set{p_1,p_2,...,p_n}$ and delivery jobs by $D = \set{d_1,d_2,...,d_n}$. The set of edges $E$ corresponds to the possible connections between jobs. For our use here all connections are possible -- that is it is possible for a vehicle to drive between any two jobs -- except where the depot is concerned. No edge terminates at $v_0$ and no edge originates at $v_{2n+1}$ and there is no edge $\edge{0,2n+1}$. Each edge $\edge{i,j} \in E$ has a corresponding cost $c_{ij}$ which typically represents the travel distance between customers.
% 
% The set of routes is denoted by $R$ and each route has a maximum capacity $q$. Each location has a demand $d_{i} \in V$. Pickup jobs have a positive demand whereas deliveries are assumed to have a negative demand. A route containing a pickup $p_i \in P$ must also contain its corresponding delivery $d_i \in D$.
% 
% The model contains the two decision variables: $X_{ij}^r$ which is defined $\forall \edge{i,j} \in E, \forall r \in R$. $X_{ij}^r$ is equal to $1$ if route $r$ contains a trip from location $v_i$ to location $v_j$ and $0$ otherwise.
% 
% Let $y_i$ denote a an intermediate variable that stores the total load of a vehicle traveling route $r$ and visiting location $v_i$.  


\begin{align}
&\intertext{Minimize:}
	&\sum_{r \in R} \sum_{\edge{ij} \in E} c_{ij}X_{ij}^r \tag{1}\label{PF:1}\\ \notag
&\intertext{Subject to:}
% general VRP constraints
	&\sum_{r \in R} \sum_{j \in V} X_{ij}^r = 1 		  		&&\quad \forall i \in V \tag{2}\label{PF:2}\\
	&\sum_{j \in V} X_{0j}^r = 1 							   	&&\quad \forall r \in R \tag{4}\label{PF:3}\\
	&\sum_{j \in V} X_{j0}^r = 1 							   	&&\quad \forall r \in R \tag{5}\label{PF:4}\\
	&\sum_{i \in V} X_{ik}^r - \sum_{j \in V} X_{kj}^r = 0 &&\quad \forall k \in C \mbox{ and } \forall r \in R \tag{6}\label{PF:5}\\
% pickup and delivery contraints
	&p_i \in r \implies d_i \in r \forall r \in R \tag{7}\label{PF:6}\\
	&p_i \text{ appears before } d_i \in r \forall r \in R \tag{8}\label{PF:7}\\
	&X_{ij}^r = 1 \implies y_i + d_i = y_j \forall r \in R \text{ and } \forall i,j \in V \tag{9}\label{PF:8}\\
	y_0 = 0 \tag{10}\label{PF:9}\\
	&\sum_{r \in R} \sum_{j \in V} X_{ij}^r y_i \leq q		&&\quad \forall i \in V \tag{11}\label{PF:10}
\end{align}

The objective function \eqref{PF:1} aims to minimize costs $c_{ij}$. Constraint \eqref{PF:2} states that each job can only be serviced by a single route. Constraint \eqref{PF:3} enforces the capacity constraint: each route can not exceed the maximum vehicle capacity. Constraint \eqref{PF:4} and \eqref{PF:5} ensure that each route starts at exactly 1 depot vertex and finishes at exactly one depot vertex. Constraint \eqref{PF:6} is a flow constraint that ensures that the number of vehicles entering a customer is equal to the number of vehicles leaving.

Constraint \eqref{PF:7} states that jobs pickup and delivery jobs are serviced by the same route. Constraint \eqref{PF:8} enforces that goods are picked up before being delivered. And constraints \eqref{PF:9} through \eqref{PF:11} ensure that each vehicle's capacity isn't exceeded.


% ---------------------
\subsubsection{Objectives}

\begin{itemize}
	\item Minimize between job travel distance
\end{itemize}

% ---------------------
\subsubsection{Constraints}

% =====================
% Sub Section
\subsection{\VRPTW}

% =====================
% Sub Section
\subsection{\PDPTW}


IN REAL WORLD CASE OFTEN VEHICLE NUMBER IS SET. IN THIS CASE WE WANT TO MAKE MAXIMUM USE OF THE VEHICLES THAT ARE AVAILABLE SINCE WAGES ARE PAID REGARDLESS.




